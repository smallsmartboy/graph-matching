
\section{Introduction}

One of the great benefits of the web as a useful source of hyperlinked
information comes from the careful choices made in crafting the recommendations
that link a given page to closely related pages. Even though this advantage was
identified well before the WWW was in place by Bush~\cite{Bush45}, it continues
to persist even tday. Seamless site navigation is enabled by carefully chosen
recommendations to closely related pages. Thus, the presence of recommendations
is an integral feature of several popular websites that are known to have high
user engagement resulting in long sessions. Examples range from 
entertainment sites like YouTube that recommend a set of videos on the right
for every video watched by a user, information sites like Quora that recommend
a set of related questions on every question page, and retail sites like Amazon
that recommend similar products on every product page. \vs

Offline recommendation systems~\cite{Schafer1999, Adomavicius2005, Resnick1997}
decide on a set of $d$ related candidate items for each item (or page) based 
on relevance. In this work, 
we analyze how to prune the set to $c < d$ recommendations 
such that globally we ensure that the resulting recommendation subgraph is 
`efficient' for traversal.\vs

We can represent this notion of recommendations by using a directed graph. A
vertex is an item and a directed edge $(u, v)$ is a recommendation from
$u$ to $v$. Note that, under this graph model for the above problem, if $c=1$ and we
require the chosen recommendation subgraph be strongly connected then our
problem reduces to Hamilton cycle, an NP-complete problem~\cite{CLRS2001}. \vs

\subsection{Bipartite Recommendation Subgraphs}

We can generalize the above graph recommendation problem by using a
directed bipartite graphs. We can use one partition $L$ to represent
the set of items for which we are required to suggest
recommendations. We can use the other partition $R$ to represent the
set of items that are potentially recommended. A single item
can be represented in both $L$ and $R$ if needed.  Now the input to
the problem is this directed graph $G$ where each vertex has $d$
recommendations. The output required is a subgraph $G'$ where each
vertex in $L$ has $c < d$ recommendations. The goal is to
minimize the number of vertices that have in-degree less than an
integer $a$. We introduce this as the $(c, a)$-recommendation subgraph
problem.  Note that if $a=c=1$ this is simply the problem of maximum
bipartite matching~\cite{LovaszPlummer1986}. If $a=1$ and $c > 1$,
we obtain a $b$-matching problem, that can be converted to the
usual bipartite matching problem~\cite{Gabow1983}.

{\bf Implementation Issues} Over the past few years, the first and
third authors have implemented complex graph algorithms in
cutting-edge web-technology companies including Google, Facebook and
BloomReach. There are two key hurdles in making such graph algorithms
practical. The first is that the method used must be very simple to
implement, debug and deploy. The second is that the method must scale
gracefully with larger sizes. \vs

Matching algorithms require linear memory and super-linear run-time
which does not scale well. For example, the Facebook graph has over a
billion vertices\cite{FacebookNodes} and hundreds of billions of edges. Even an
e-commerce website with 100M product pages and 100 recommendation candidates per
product would require easily over 160GB in main memory to run these
algorithms without using graph compression techniques that add additional technical difficulties
in development and maintenance. In practice offline problem instances are solved by using
distributed computing such as map-reduce~\cite{DeanGhemawat2004}. 
However, efficient map-reduce algorithms for graph problems
are notoriously difficult and we are back to building a complicated
system. \vs

% http://www.longtail.com/the_long_tail/faq/
% The long tail: Why the future of business is selling less of more
% Growth dynamics of the World-Wide Web. Nature - Long tail by page
% Analysis of WWW traffic in Cambodia and Ghana - Zipf by page
% Location and time do matter: A long tail study of website requests: http://public.csusm.edu/ckumar/research/Kumar-etal-online-location-time-DSS.pdf 
% shows % Zipf with beta=-1.365
%%

\subsection{Algorithms and Analyses}

{\bf The Lazy Engineer} These reasons motivated the authors to
investigate the ``lazy" approach of choosing a very simple (almost
any) set of recommendation to see if they would produce near optimal
solutions at least under realistic scenarios in practice.  In
practice, this would immediately imply that a very simple back-end
system can be built in many cases without the need for a complicated
algorithm. If the thresholds provided by our analysis are insufficient
in quality, then the software designer can consider implementing the classical
algorithms. \vs

To be more specific about practical scenarios we briefly describe
web graph characteristics to give an idea for the sizes of $L$,
$R$, $c$ and $a$ to expect. It is well known, and can be verified by
prior experiences of the authors that a small number of `head' pages contribute to a
significant amount of the traffic while a long tail of the remaining
pages contribute to the rest~\cite{HubermanAdamic1999, DuDemmerBrewer2006}. As
demonstrated by one of the prior measurements~\cite{KumarNorrisSun2009}
it is not unreasonable to expect that 50\% of the site traffic is
contributed by less than 1\% (a few thousands) of the web pages and a
large number of tail pages (a few hundreds of thousands) contribute
the other half of the traffic. This implies that in practice $L$ can
be up to two orders of magnitude smaller than $R$.  By observing
recommendations of Quora, Amazon, YouTube and our work in building
these technologies in practice, typical values for $c$ range from 3 to
20 recommendations per page. Values of $a$ are the hardest to nail
down but one can think of it as something that can range from $1$ to
$10$.


{\bf Recommendation Graph Models} The lazy engineer's algorithm
essentially picks a random $c$ out of $d$ edges. The natural input graph model to
study this is therefore the {\em
 fixed-degree} model where every
node on the left has recommendations to a random subset of size $d$
nodes in the right. This model is similar to the another model called
$d$-out, where vertices from both sides of the bipartition send $d$
edges to the other side of the bipartition \cite{FriezePittel2004}. We
study this model in Section~\ref{fixed-degree}. Our main result
identifies the range of parameters involving $c,a,l=|L|$ and $r =|R|$
where the lazy algorithm is very effective. In addition to showing
that it is a $(1-\frac1e)$- approximation algorithm in expectation,
we
 also get much better bounds for the expected performance for a
wide
 range of realistic parameters and also prove high probability
bounds
 on its performance. We extend the models later to more
realistic
 graphs including a hierarchical, cartesian and weighted
graph
 models. \vs


{\bf The Greedy Algorithm} While the lazy algorithm chooses any set of $c$ recommendations, the 
greedy algorithm will need some work: it scans the nodes on the right that must
be discovered, and checks if there are $a$ neighbors from the left that have
not exhausted their bound of $c$ edges in the subgraph, and if so, it uses them to
add this node to the discovered set. We study this algorithm in Section~\ref{greedy}. 
We show the easy result that Greedy gives a $\frac{1}{a+1}$-approximation, 
and also do an average case analysis. We
use the usual Erd\"os-Renyi model\cite{ErdosRenyi59} for the analysis.
%In the subsequent Section~\ref{worst-vs-avg}, we compare and plot the worst-case
%performance of Greedy against the average case expected performance of the random
%or lazy algorithm, as a basis for our later computational comparisons in a
%similar vein. \vs

{\bf Optimal Recommendation Subgraphs}
Finally we turn to the question of whether the maximum number of nodes allowed
by the degree constraints can be covered (at least $a$ times) by a recommendation
subgraph: we say that there is an perfect $(c,a)$-recommendation subgraph in this
case. Under the usual Erd\"os-Renyi model, we use prior results on the
existence of a perfect matching to characterize the edge probability over which
there exist such optimal recommendation subgraphs, by using a subset partitioning
method for the analysis. By relaxing the condition of optimality of matching (i.e.
perfect matchings) to disallowing short augmenting paths in these methods, we also
turn the above results into $(1-\epsilon)$-approximation algorithms trading off
the running time, for appropriately dense random graphs as before. \vs
