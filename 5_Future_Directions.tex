\section{Conclusions}

In this paper we proposed several different models for explaining how
recommendation subgraphs arise probabilistically and how optimization
problems on these graphs can be solved, but there is much more that
can be done both on the theoretical and the empirical fronts. On the
theory side the biggest open problem is the hardness of the general
$(c, a)$-recommendation subgraph problem. For specific values of $c$
and $a$ we know polynomial time algorithms. For e.g., $a=1$ leads to
either bipartite matching or $b$-matching which can be solved in
polynomial time but the hardness of the general problem remains
open. While initial experiments are promising, extensive empirical
analysis using real data can be revealing in terms of goodness of fit
of the models and the performance of the algorithms.

\subsection*{Acknowledgments}
The authors would like to thank Alan Frieze for helpful discussions.

%On the empiral side, the graph models that we introduced can be
%made richer. For example, the hierarchical graphs require the trees to
%be balanced and of the same size which is constraining. The edge
%weights on the weighted graph is sampled from a distribution as
%opposed to being arbitrary. We expect the research community and our
%future work to follow-up along both directions in the future.

%\begin{enumerate}

%\item We do not know of a polynomial time algorithm that can solve the
%  $(c, a)$-recommendation subgraph problem. For specific values of $c, a$ we
%  know polynomial time algorithms. For e.g., $a=1$ leads to either
%  bipartite matching or fractional matching which can be solved in
%  polynomial time. 

%\item In practice, there might exist several different
%  recommendation subgraphs based on different features. For example,
%  two people might be related because they went to the same 
%  school, or because they live in the same city, or because they
%  would complete a large number of triangles, etc. It's worthwhile
%  to investigate how such graphs can be combined into one, or how
%  an optimization problem can be solved using all such 
%  recommendation subgraphs simultaneously.

%\item We should devise metrics that can evaluate how well a
%  recommendation subgraph fits a given model and conduct some parameter
%  fitting experiments to see how well actual recommendation subgraphs
%  which arise in practice fit our models.

%\item We should implement the sampling and the greedy algorithms
%  given in the paper to see if they can solve to near optimality
%  the graph recommendation problems that arise in practice.

%\end{enumerate}
