\subsection{Existence of Perfect Recommendation Subgraphs}
Define a \emph{perfect} $(c,a)$-recommendation subgraph on $G$ to be a subgraph $H$ such that
$deg_H(u)\leq c$ for all $u\in L$ and $deg_H(v)=a$ for
$\min(r,cl/a)$ of the vertices in $R$. In this section we will
prove a sufficient condition for perfect $(c,a)$-recommendation
subgraphs to exist in a bipartite graph $G$ under the Erd\"os-Renyi model\cite{ErdosRenyi59} where edges are sampled
uniformly and independently with probability $p$.
We use the algorithm we propose to prove this condition as a candidate to compare against random choice and greedy in our tests.
Our result relies on
the following characterization of perfect matchings.

\begin{thm}\cite{Janson2011}
\label{random_matching_threshold}
Let $G$ be a bipartite graph drawn from $G_{n, n, p}$. If $p \geq \frac{\log n -
\log\log n}{n}$, then as $\lim_{n\to\infty}$ probability that G has a perfect
    matching approaches 1.
\end{thm}

We will prove that a perfect $(c,a)$-recommendation subgraph exists in
random graphs with high probability by building it up from $a$
matchings each of which must exist with high probability if $p$ is
sufficiently high. In particular, we show that $p$ only needs to
be $\Omega(\frac{\log n}{n})$ for this to succeed.
%While the theorem is stated for
%the case when $a \leq c$, it applies equally well to the $a > c$
%case by partitioning $L$ instead of $R$ in the following proof.

\begin{thm}
Let $G$ be a random graph drawn from $G_{l, r, p}$ with $p\geq a\frac{\log l-\log\log
l}{l}$ and $kc \geq a$, then the probability that $G$ has a perfect $(c, a)$-recommendation
subgraph tends to 1 as $l,r\to\infty$.
\end{thm}

\begin{proof}
Given the size and the degree constraints of $L$, at most $lc/a$
vertices in $R$ can have degree $a$ in a $(c,a)$-recommendation subgraph. We
therefore restrict $R$ to an arbitrary subset $R'$ of size $lc/a$.
Next, we pick an enumeration of the vertices in $R'=\{v_0,\ldots, v_{lc/a-1}\}$
and add each of these vertices into $a$ subsets as follows. Define
$R_i = \{v_{(i-1)l/a}, \ldots, v_{(i-1)l/a+l-1}\}$ for each $1\leq i\leq c$ where
the arithmetic in the indices is done modulo $lc/a$. Note both $L$ and all of
the $R_i$'s have size $l$. \vs

Using these new sets we define the graphs $G_i$ on the bipartitions
$(L, R_i)$. Since the sets $R_i$ are intersecting, we cannot define the
graphs $G_i$ to be induced subgraphs. However, note that each vertex $v\in R'$
falls into exactly $a$ of these subsets. Therefore, we can uniformly randomly assign each edge in $G$ to one of $a$ graphs among $\{G_1,\ldots, G_c\}$ it can fall into,
and make each of those graphs a random graph. In fact, while the different
$G_i$ are coupled, taken in isolation we can consider any single $G_i$ to be
drawn from the distribution $G_{l,l,p/a}$ since $G$ was drawn from $G_{l,r,p}$.
Since $p/a \geq (\log l - \log\log l)/l$ by assumption, we conclude by
Theorem~\ref{random_matching_threshold}, the probability that a particular
$G_i$ has no perfect matching is $o(1)$. \vs

Considering $c$ to be fixed, by a union bound, we then conclude that except
for a $o(1)$ probability, each one of the $G_i$'s has a perfect matching. By
superimposing all of these perfect matchings, we can see that every vertex in
$R'$ has degree $a$. Since each vertex in $L$ is in exactly $c$ matchings, each
vertex in $L$ has degree $c$. It follows that except for a $o(1)$ probability
there exists a $(c,a)$-recommendation subgraph in $G$.
\end{proof}

\subsection{Approximation Algorithm Using Perfect Matchings}

The above result now enables us to design a near linear time
algorithm with a $(1-\epsilon)$ approximation guarantee
to the $(c,a)$-recommendation subgraph problem by leveraging
combinatorial properties of matchings.

\begin{thm}
Let $G$ be drawn from $G_{l,r,p}$ where $p \geq a\frac{\log l - \log\log l}{l}$. Then there
exists an algorithm that can find a $(1-\epsilon)$-approximation in time
$O(\frac{|E|c^2}{\epsilon})$ with probability $1-o(1)$.
\end{thm}
\begin{proof}
As with the proof of the previous theorem, we can arbitrarily restrict $R$ to a
subset of size $lc/a$ and divide $R$ into sets $R_1,\ldots,R_c$ such that each
vertex in $R$ is contained in exactly $a$ of the sets. Using the previous
theorem, we know that each of the graphs $G_i$ has a perfect matching with high
probability. These perfect matchings can be approximated to a $1-\epsilon/c$
factor by finding matchings that do not have augmenting paths of length
$\geq 2c/\epsilon$~\cite{LovaszPlummer1983}. This can be done for each
$G_i$ in $O(|E|c/\epsilon)$ time. Furthermore, the union of unmatched vertices
makes up an at most $c(\epsilon/c)$ fraction of $R'$.
\end{proof}

