\section{Summary and Future Work}
We have presented a new class of structural recommendation problems
cast as subgraph selection problems, and analyzed three algorithmic
strategies to solve these problems because graph matching algorithms can be
prohibitive to implement in real-world scenarios. The sampling method is most
efficient, the greedy approach trades off computational cost with
quality, and the partition method is effective for smaller problem
sizes. We have proved effective theoretical bounds on the quality
of these methods, and also substantiated them with experimental
validation both from simulated data as well as from real data from
retail web sites. Our findings have been very useful in the
deployment of effective structural recommendations in web relevance
engines that drive many of the leading online websites of popular
retailers. \vs

%The simple sampling method and its analysis extends to more general
%models of random graphs: in one version, we can consider
%hierarchical models that take into account the product hierarchy
%trees under which the pages in $L$ and $R$ are situated. A second
%version considers a Cartesian product model where the pages in $L$
%and $R$ are partitioned into closely related blocks and the graph
%induced between every pair of left-right blocks follows a fixed
%degree random model. A third variant models the potential flow of
%customer traffic over each possible recommended edge from a left to
%right page with nonnegative weights, and the resulting problem is
%to find a subgraph where the number of right nodes with at least a
%certain minimum amount of recommended traffic. Validating these
%more general models by fitting real life data to them as well as
%corroborating the performance of various methods in simulated and
%real data for these models could yield an even better understanding
%of our suggested algorithmic strategies for the structural
%recommendation subgraph problem.\vs

{\bf Acknowledgement:} We thank Alan Frieze and Ashutosh Garg for helpful
discussions.
