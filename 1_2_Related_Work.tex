\section{Related Work}

Recommendation systems have been studied extensively in literature, espcially since the advent of the web. Most recommendation systems can be broadly separated into two different groups: collaborative filtering systems and content-based recommender systems \cite{almazro2010survey}. Much attention has been focused on the former approach, where either users are clustered by considering the items they've consumed or items are clustered by considering the users that have bought them. Both item-to-item and user-to-user recommendation systems based on collaborative filtering have been adopted by many industry giants such as Twitter \cite{twitter-collab-filtering}, Amazon \cite{amazon-collab-filtering} and Google \cite{google-collab-filtering}. 

Content based systems instead look at each item and its intrinsic properties. For example, Pandora has categorical information such as Artist, Genre, Year, Singer, Tempo etc. on each song it indexes. This categorical data can then be used to recommend new songs that are similar to the songs that a user has liked before. Depending on user feedback, a recommender system can learn which of the categories are more or less important to a user and adjust its recommendations.

A drawback of the first type of system is that is that they require multiple visits by many users so that a taste profile for each user, or a user profile for each item can be built. This is possible for large commerce or entertainment hubs such as the companies mentioned above, but not very likely for arbitrary online retailers. Content-based systems are unsuitable for our purposes as well, because they also require significant user participation to train the underlying system. These conditions are possible to meet for large commerce or entertainment hubs such as the companies mentioned above, but not very likely for most online retailers that specialize in a just a few areas.

Because of this constraint, in this paper we focus on a recommender system that uses many different algorithms that extract categorical data from item descriptions and uses these to establish weak links between items. In the absence of data that would enable us to choose among these many links, we consider every potential recommendation to be of equal value and focus on the objective of discovery, which has not been studied before.

% we need to add the long tail here somewhere but I'm not sure how.