\section{Algorithms for Recommendation Subgraphs}
In this section, we analyze the lazy algorithm of choosing any set of
$c$ recommendations, and the slightly more interesting greedy algorithm
for finding a $(c,a)$-recommendation subgraph. We begin by introducing
the fixed-degree random graph model for the input supergraph.


\subsection{Fixed Degree Model}
\label{fixed-degree}

In this model, we assume that a bipartite graph $G=(L,R,E)$ is
generated probabilistically as follows. Each vertex $v\in L$
uniformly samples a set of $d$ neighbors from $R$. The solution subgraph $H$
is now sampled from $G$ by uniformly sampling $c\leq d$ edges incident
on each vertex. Let $|L|=l$, $|R|=r$ and $k=l/r$.
The following theorem gives a lower bound on the expected solution.


%\begin{thm}\label{original_result}
%Suppose that $G=(L,R,E)$ and $H\subseteq G$ is generated as above. Then
%\[ \E[S] \geq r(1-\exp(-ck))\]
%where the expectation is over the random sampling of $G$ and $H$.
%\end{thm}
%\begin{proof}
%For each $v\in R$ let $X_v$ be the indicator variable for the event
%that $\deg_H(v) \geq 1$. Note that since for each vertex $u\in L$, $H$
%uniformly samples from a uniformly sampled set of neighbors, we can
%think $H$ as being generated by the same process that generated $G$,
%but with $d$ replaced with $c$. Now for a specific vertex $u \in R$,
%the probability that it has no incident edges is
%$\left(1-\frac{1}{r}\right)^c$. Since the selection of neighbors for each
%vertex in $L$ is independent, it follows that that:
%\[ \Pr[X_v=0] = \left(1-\frac{1}{r}\right)^{cl} \leq \exp\left(-c \cdot \frac{l}{r}\right) = \exp(-ck) \]
%Note that $S = \sum_{v\in R} X_v$. Applying linearity of expectation, we get
%\[ \E[S] = \sum_{v\in V} \E[X_v] \geq r(1-\exp(-ck))\]
%\end{proof}

%While this shows a lower bound in absolute terms, we must compare it to the best possible solution {\em OPT}. The follow theorem proves the approximation ratio to {\em OPT}.


\begin{thm}\label{original_result}
Let $S$ be the
random variable denoting the number of vertices $v \in R$ such that
$\deg_{H}(v)\geq a$ in the fixed-degree model. Then
\[ \emph{\E}[S] \geq r\left(1-e^{-ck+\frac{a-1}{r}}\frac{(ck)^a-1}{ck-1}\right)  \]
\end{thm}

\begin{proof}
Let $X_{uv}$ be the indicator variable of the event that the edge $uv$
($u\in L$, $v\in R$) is in the subgraph that we picked
and set $X_{v} = \sum_{u\in L} X_{uv}$ so that $X_{v}$ represents the
degree of the vertex $v$ in our subgraph. Because our algorithm
uniformly subsamples a uniformly random selection of edges, we can
assume that $H$ was generated the same way as $G$ but sampled $c$
instead of $d$ edges for each vertex $u\in L$. So $X_{uv}$ is a
Bernoulli random variable. Using the bound $\binom{n}{i}
\leq n^i$ on binomial coefficients we get,
\begin{align*}
      \Pr[X_v < a]
&=    \sum_{i=0}^{a-1} \binom{cl}{i} \left(1-\frac{1}{r}\right)^{cl-i}\left(\frac{1}{r}\right)^i
 \leq \sum_{i=0}^{a-1} \left(\frac{cl}{r}\right)^i\left(1-\frac{1}{r}\right)^{cl-i} \\
&\leq    \left(1-\frac{1}{r}\right)^{cl-(a-1)} \cdot \sum_{i=0}^{a-1} (ck)^i
 \leq \left(1-\frac{1}{r}\right)^{cl-(a-1)}\frac{(ck)^a-1}{ck-1} \\
&\leq e^{-ck+\frac{a-1}{r}} \frac{(ck)^a-1}{ck-1}
\end{align*}


Letting $Y_v = \left[X_v \geq a\right]$, we now see that

\[ \E[S] = \E\left[\sum_{v\in R} Y_v\right] \geq r\left(1-e^{-ck+\frac{a-1}{r}} \frac{(ck)^a-1}{ck-1}\right) \]
\end{proof}

We can combine this lower bound with a trivial lower bound to obtain an
approximation ratio that holds in expectation.

\begin{thm}
The above sampling algorithm gives a $\left(1-\frac1e\right)$-factor approximation to the $(c,1)$-graph recommendation problem in expectation.
\end{thm}
\begin{proof}
The size of the optimal solution is bounded above by both the number
of edges in the graph and the number of vertices in $R$. The former of
these is $cl=ckr$ and the latter is $r$, which shows that the optimal solution size
$OPT \leq
r\max(ck,1)$. Therefore, by simple case analysis the approximation ratio
in expectation is at least
$ ({1-\exp(-ck)})/\min(ck,1) \geq 1-\frac{1}{e} $
\end{proof}


For the $(c, 1)$-recommendation subgraph problem the approximation obtained by this sampling
approach can be much better for certain values of $ck$. In particular,
if $ck>1$, then the approximation ratio is $1-\exp(-ck)$, which
approaches 1 as $ck\to\infty$. In particular, if $ck=3$, then the
solution will be at least 95\% as good as the optimal solution even
with our trivial bounds. Similarly, when $ck<1$, the approximation
ratio is $(1-\exp(-ck))/ck$ which also approaches 1 as $ck\to 0$. In
particular, if $ck=0.1$ then the solution will be at 95\% as good as
the optimal solution. The case when $ck=1$ represents the
worst case outcome for this model where we only guarantee 63\%
optimality. Figure~\ref{fig:simple_approx} shows the approximation ratio as a
function of $ck$.\vs

\begin{figure}[b]
\centering
\begin{minipage}[h]{0.45\textwidth}
  \centering
  \includegraphics[width=0.8\textwidth]{images/sri_Original.png}
  \caption{Approx ratio as a function of $ck$ }\label{fig:simple_approx}
\end{minipage}
\hspace{0cm}
\begin{minipage}[t]{0.45\textwidth}
  %\vspace{.2cm}
  \centering
  \begin{tabular}{ |c|c|c|c|c|c| }
    \hline
    $a$ & 1 & 2 & 3 & 4 & 5 \\ \hline
    $ck$ & 3.00 & 4.74 & 7.05 & 10.01 & 13.48 \\
    \hline
  \end{tabular}
  \vspace{0.7cm}
  \caption{The required $ck$ to obtain 95\% optimality for $(c, a)$-recommendation subgraph}
  \label{a-values}
\end{minipage}
\end{figure}



%Now suppose that $G$ is generated and $H$ is sampled using the same
%processes as described above. The next theorem, extends the above
%bounds to the $(c,a)$-graph recommendation problem where $a>1$.
%In particular if we set $a=1$, we will obtained the estimate from the original analysis.

%Now, we can perform a similar analysis as before. In
%particular,
For the general $(c, a)$-recommendation subgraph problem, if $ck>a$, then the problem is easy on average. This
is in comparison to the trivial estimate of $cl$. For a fixed $a$, a
random solution gets better as $ck$ increases because the decrease in
$e^{-ck}$ more than compensates for the polynomial in $ck$ next to
it. However, in the more realistic case $ck<a$, we need to use the
trivial estimate of $ckr/a$, and the analysis for $a=1$ does not extend here.
The table in Figure~\ref{a-values} shows how large
$ck$ needs to be for the solution to be 95\% optimal for different
values of $a$.\vs

%In both this analysis and the previous one, $ck$ is the average degree
%of a vertex $v\in R$ in our chosen subgraph. The original analysis
%showed that if $ck>1$, then the sampling algorithm will probably cover
%every vertex in $R$ since the expected degree of each vertex is large.
%On the other hand if $ck$ is small ($ck < 1$) then the best possible
%solution is obtained when none of the vertices in $R$ has degree
%greater than 1. \vs

%If $ck<1$, then we do not cover very
%many vertices in $R$, but we also do not cover many vertices more than
%once. Since the optimal solution in this case was correspondingly low,
%our solution was good in the $a=1$ case. However, when $a<1$, the fact
%that our edges are well-dispersed only hurts our solution because we
%need to concentrate the edges on specific nodes in $R$ that will
%eventually count in the objective.


We close out this section by showing that the main result that holds in expectation also hold with high probability for $a=1$, using the following variant of Chernoff bounds.
%While Chernoff bounds are usually stated for independent variables, the
%variant below holds for any number of pairwise non-positively correlated
%variables.

\begin{thm}\label{negative_corr_chernoff}~\cite{AugerDoerr2011}
Let $X_1,\ldots, X_n$ be non-positively correlated variables. If $X=\sum_{i=1}^n X_i$, then for any $\delta\geq 0$
\[ \Pr[X \geq (1+\delta)\emph{\E}[X] ] \leq \left(\frac{e^\delta}{(1+\delta)^{1+\delta}}\right)^{\E[X]} \]
\end{thm}

%Using this we can convert our expectation result to one that holds
%with high probability.

\begin{thm}
Let $S$ be the random variable denoting the number of vertices $v \in R$ such that $\deg_{H}(v)\geq 1$. Then
$ S \leq r(1-2\exp(-ck))$ with probability at most $(e/4)^{r(1-\exp(-ck))}$.
\end{thm}

\begin{proof}
We can write $S$ as $\sum_{v\in R} 1-X_v$ where $X_v$ is the indicator
variable that denotes that $X_v$ is matched. Note that the variables
$1-X_v$ for each $v\in R$ are non-positively correlated. In
particular, if $N(v)$ and $N(v')$ are disjoint, then $1-X_v$ and
$1-X_{v'}$ are independent. Otherwise, $v$ not claiming any edges can
only increase the probability that $v'$ has an edge from any vertex
$u\in N(v)\cap N(v')$. Also note that the expected size of $S$ is
$r(1-\exp(-ck))$ by Theorem \ref{original_result}. Therefore, we can
apply Theorem \ref{negative_corr_chernoff} with $\delta=1$ to obtain
the result.
\end{proof}
For realistic scenarios where $r$ is very large, this gives very good bounds. 