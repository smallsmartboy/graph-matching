\abstract


Recommendations are central to the utility of many popular e-commerce
websites. Such sites typically contain a set of recommendations on
every product page that enables visitors and crawlers to easily
navigate the website.  These recommendations are essentially universally present
on all e-commerce websites. Choosing an appropriate set of recommendations
at each page is a critical task performed by dedicated backend
software systems. \vs

We formalize the concept of recommendations used for discovery as a natural graph optimization
problem on a bipartite graph: the left partition represent highly visited pages while the right 
represents rarely visited ones, and the edges are candidate recommendation links that can be used. 
The goal is to pick at most a fixed number of out-links from each left node to maximize the number of 
right nodes that are in-linked redundantly. The allowed out-degree of left nodes and minimum redundancy
 of coverage of right nodes are parameters defining the problem.

We propose three methods for solving the problem in increasing order
of sophistication: a local random sampling algorithm, a greedy algorithm
and a more involved partitioning based algorithm. 
We first theoretically analyze the performance of these three methods
on random graph models and characterize when each method will yield a
solution of sufficient quality and the parameter ranges when more
sophistication is needed. We complement this by providing an empirical
analysis of these algorithms on simulated and real-world production
data from a retail website. Our results confirm that it is not always necessary to implement  complicated algorithms in the real-world, and
demonstrate that very good practical results can be obtained by
using simple heuristics that are backed by the confidence of concrete
theoretical guarantees. \vs

\iffalse

We formulate the problem of choosing the short list of recommendations from each page as one of choosing a subgraph of the candidate recommendation graph with the objective of maximizing the size of the recommended pages.

To inter-connect the website for efficient
traversal it is critical to choose a few recommendations on each
page from the list of all candidates.

In our formulation of the problem, the set of pages on the site is
divided into a few popular pages where surfers arrive and the
remaining large number of pages that should be recommended from the
popular pages. The natural bipartite graph of potential
recommendations is a candidate supergraph. Given the limited space to
display recommendations a subgraph that has bounded outdegree $c$ on the
popular pages must be chosen. Our goal is to maximize the number of
undiscovered pages that have indegree at least some target $a$. We introduce and study this
as the $(c, a)$-recommendation subgraph problem.

Solving such problems optimally at web-scale typically involves
writing distributed matching algorithms.  These can be incredibly hard
to implement, debug and scale effectively.  Instead, in this work, we
study the effectiveness of a lazy engineer solving the recommendation
subgraph problem. We investigate the cases when the candidate
supergraph is a random graph under two models: the fixed-degree model
where every node on the left has exactly $d$ random neighbors on the
right, as well as the standard Erd\"{o}s-Renyi model with expected
degree $d$ on the left side. We show that for most reasonable
parameters of the models, the lazy engineer would have found solutions
very close to optimal. We further show the conditions under which a
perfect recommendation subgraph, a generalization of perfect
matchings, exists. Lastly, to more realistically model web graphs, we
propose generalizations of the random graph models using topic
taxonomies, varying subgraph edge densities and edge weights that
capture the strength of recommendations. Surprisingly, the lazy
engineer would still have found near optimal solutions under different
realistic parameters, which we validate with some computational
testing on simulated models.
\fi 
