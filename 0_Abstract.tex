\abstract

Recommendations are central to the utility of many of the popular
websites such as YouTube, Facebook and Quora, as well as many
e-commerce store websites. Such sites usually have a large list of
candidate pages that could serve as recommendations from the current
page based on relevance. To inter-connect the website for efficient
traversal it is critical to choose a few recommendations on each
page from the list of all candidates.

In our formulation of the problem, the set of pages on the site is
divided into a few popular pages where surfers arrive and the
remaining large number of pages that should be recommended from the
popular pages. The natural bipartite graph of potential
recommendations is a candidate supergraph. Given the limited space to
display recommendations a subgraph that has bounded outdegree $c$ on the
popular pages must be chosen. Our goal is to maximize the number of
undiscovered pages that have indegree at least some target $a$. We introduce and study this
as the $(c, a)$-recommendation subgraph problem.

Solving such problems optimally at web-scale typically involves
writing distributed matching algorithms.  These can be incredibly hard
to implement, debug and scale effectively.  Instead, in this work, we
study the effectiveness of a lazy engineer solving the recommendation
subgraph problem. We investigate the cases when the candidate
supergraph is a random graph under two models: the fixed-degree model
where every node on the left has exactly $d$ random neighbors on the
right, as well as the standard Erd\"{o}s-Renyi model with expected
degree $d$ on the left side. We show that for most reasonable
parameters of the models, the lazy engineer would have found solutions
very close to optimal. We further show the conditions under which a
perfect recommendation subgraph, a generalization of perfect
matchings, exists. Lastly, to more realistically model web graphs, we
propose generalizations of the random graph models using topic
taxonomies, varying subgraph edge densities and edge weights that
capture the strength of recommendations. Surprisingly, the lazy
engineer would still have found near optimal solutions under different
realistic parameters, which we validate with some computational
testing on simulated models.
