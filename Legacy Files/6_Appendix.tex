\section{Proofs of More General Models}\label{sec:appendix}

\setcounter{thm}{9}
We prove the main theorem about hierarchical models from Section~\ref{hierarchy}.
\begin{thm}
Let $S$ be the subset of edges $v\in R$ such that $\deg_H(v) \geq 1$ in the hierarchical tree model. Then
\[ \E[S] \geq r(1-\exp(-ck)) \]
\end{thm}

\begin{proof}
Let $v\in R$ and let $T_L^{D-1}, T_L^{D-2}\backslash T_L^{D-1},
\ldots, T_L^0\backslash T_L^1$ be the sets it can take edges
from. Since $T_L$ and $T_R$ split perfectly evenly at each node the
vertices in these sets will be chosen from $r_{D-1}, r_{D-1},
r_{D-2},\ldots, r_{1}$ vertices in $R$ as neighbors,
where $r_i$ is the size of subtree of the right tree rooted at depth
$i$. Furthermore, each of these sets described above have size
$l_{D-1}, l_{D-1}, l_{D-2}, \ldots, l_{1}$ respectively, where $l_i$
is the size of a subtree of $T_L$ rooted at depth $i$. It follows that
the probability that $v$ does not receive any edges at all is at most

\begin{align*}
	      \Pr[\lnot X_v]
	&=    \left(1-\frac{1}{r_{D-1}}\right)^{c_0l_{D-1}}\prod_{i=1}^{D-1}\left(1 - \frac{1}{r_i}\right)^{c_{D-i} l_i} \\
	&\leq \exp\left(-\frac{l_{D-1}}{r_{D-1}}c_0\right)\prod_{i=1}^{D-1} \exp\left(-\frac{l_i}{r_i}c_{D-i}\right) \\
	&=    \exp\left(-(c_0 + \ldots + c_{D-1})k\right) \\
	&=    \exp(-ck)
\end{align*}

Since this is an indicator variable, it follows that
\[ \E[S] = \E\left[\sum_{v \in R} X_v \right] \geq r \left(1-\exp(-ck)\right) \]
\end{proof}

Next, we prove our main bound on the random choice solution under the cartesian product model introduced in Section~\ref{cartesian}.

\begin{thm}
Let $S$ be the subset of edges $v\in R$ such that $\deg_H(v) \geq 1$ in the cartesian product model. Then
%\vspace{-0.1in}
\[ \E[S] \geq r - \sum_{j=1}^{t'} r_j \exp\left(-\sum_{i=1}^t c_{ij} \frac{l_i}{r_j}\right)\]
\end{thm}
\begin{proof}
Let $v_j \in R_j$ be an arbitrary vertex and let $X_{v_j}$ be the
indicator variable for the event that $\deg_H(v_i) \geq 1$. The
probability that none of the neighbors of some $u_i\in R_i$ is $v_j$
is exactly $(1-\frac{1}{r_j})^{c_{ij}}$. It follows that the
probability that the degree of $v_j$ in the subgraph $H[L_i,R_j]$ is 0
is at most $(1-\frac{1}{r_j})^{c_{ij}l_i}$. Considering this
probability over all $R_j$ gives us:
\[ \Pr[X_{v_i} = 0] = \prod_{i=1}^{t} \left(1-\frac{1}{r_j}\right)^{c_{ij} l_i} \leq \exp\left(-\sum_{i=1}^t c_{ij} \frac{l_i}{r_j}\right)\]

By linearity of expectation $\E[S] = \sum_{i=1}^{t'} r_i \E[X_{v_i}]$,
so it follows that
\[ \E[S] \geq \sum_{j=1}^{t'} r_j \left(1-\exp\left(-\sum_{i=1}^t c_{ij} \frac{l_i}{r_j}\right)\right) = r - \sum_{j=1}^{t'} r_j \exp\left(-\sum_{i=1}^t c_{ij} \frac{l_i}{r_j}\right)\]
\end{proof}

Finally, we give the proof of the performance of the random choice algorithm of the lazy engineer for the weighted model from Section~\ref{weighted}.
\begin{thm}
Let $G=K_{l,r}$ be a complete bipartite graph where the edges have i.i.d. weights and come from a distribution with mean $\mu$ that is supported on $[0,b]$; Assume that $ck\mu \geq 1+\epsilon$ for some $\epsilon > 0$. If the algorithm from Section \ref{fixed-degree} is used to sample a subgraph $H$ from $G$ and $S$ is the set of vertices in $R$ of incident weight at least one, then
\[ \E[S] = \sum_{v\in R} \E[X_v] = r\left(1-\exp\left(-\frac{2l\epsilon^2}{b^2}\right)\right) \]
\end{thm}

\begin{proof}
For each edge $uv\in G$, let $W_{uv}$ be its random weight, $Y_{uv}$ be
the indicator for the event $uv\in H$ and define $X_{uv} = Y_{uv}
W_{uv}$. Since weights and edges are sampled by independent processes,
we have $\E[X_{uv}] = \E[W_{uv}]\E[Y_{uv}]$ for all edges. Since $c$
edges out of $r$ are picked for each vertex, $\E[Y_{uv}] = \frac{c}{r}$
, so $\E[X_{uv}] = \frac{c}{r}\mu$. Therefore, the expected weight
coming into a vertex $v\in R$ would be
\[ \E[X_v] = \sum_{u\in L} \E[X_{uv}] = \frac{cl\mu}{r} = ck\mu\]

However, $X_{uv}$ for each $u$ are i.i.d random variables. Since by
assumption $ck\mu = 1+\epsilon$, by Hoeffding bounds~\cite{Hoeffding1963}
we can obtain
\[ \Pr[X_v \leq 1] = \Pr[X_v - \E[X_v] \geq \epsilon] \leq \exp\left(-\frac{2l\epsilon^2}{b^2}\right) \]

By linearity of expectation we can now get the result in the theorem
\[ \E[S] = \sum_{v\in R} \E[X_v] = r\left(1-\exp\left(-\frac{2l\epsilon^2}{b^2}\right)\right) \]
\end{proof}
We make two remarks about this proof. The first is that
since the variables $X_v$ are negatively correlated, our results in
Subsection \ref{fixed-degree} can be extended to the results of this
section. The second is that the condition that $W_{uv}$ are i.i.d
is not necessary to obtain the full effect of the analysis. Indeed,
the only place in the proof where the fact that $W_{uv}$ are i.i.d
is when we argued that $X_{uv}$ is large with high probability by a
Hoeffding bound. For the bound to apply, it is sufficient to assume
that $W_{uv}$ for all $v$ are independent. In particular, it is
possible that $W_{uv}$ for all $u$ are inter-dependent. This allows
us to assume a weight distribution that depends on the strength of
the recommender and the relevance of the recommendation separately.
